\chapter{PRNG (Find en god titel til kapitlet)}
Før der introduceres til pseudotilfældige tal og genereringen af disse, er det vigtigt at vide og forstå, hvad tilfældige tal egentlig er, og hvilke anvendelsesmuligheder de har. Når der tænkes på tilfældige tal, er det svært at argumentere for, om et bestemt tal er tilfældigt eller ej. For eksempel giver det ikke mening at overveje, om tallet 9 er tilfældigt. Derimod giver det god mening at tænke på en sekvens af tal, hvor tallene i sekvensen er valgt tilfældigt. At blive valgt tilfældigt betyder, at hvert tal vælges uafhængigt af alle de andre tal i sekvensen, og at sandsynligheden for at vælge et tal er ens for alle de mulige tal. Tallene følger altså en uniform fordeling \textbf{(reference til kapitlet om uniformfordelingen)}. Med udgangspunkt i cifrene 0 til 9, betyder det at hvert cifre vil blive valgt $\frac{1}{10}$
af tiden.Anvendelsen af tilfældige tal er vidtrækkende og rækker over flere forskellige brancher. I sin bog “The Art of Computer Programming” \cite{Knuth1998}
giver professor Donald E. Knuth fra Stanford University eksempler på nogle af de mange anvendelser af tilfældige tal, der understreger deres vigtighed. Disse inkluderer tekniske anvendelser som anvendelse af tal til computersimulering af virkelige naturfænomener, men også helt anderledes anvendelser som til æstetiske formål, hvor lidt tilfældighed eksempelvis er med til at gøre computer-genererede grafikker mere livlige.
