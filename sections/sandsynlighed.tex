\section{Sandsynlighed}
Noget indledende tekst om sandsynlighed

\subsection{Udfaldsrum og hændelser}
Mængden af alle mulige udfald fra et statisk eksperiment kaldes for \emph{udfaldsrummet,  S}, et hvert udfald kalde et element eller medlem af udfaldsrummet. Det vil sige, at udfaldsrummet for kast med en mønt indeholder de to elementer; plat og krone.
\newline
En delmængde af udfaldsrummet kaldes en \emph{hændelse, A}, dette noteres som $A \subseteq S$. Ved kast med en terning kan en hændelse, $A$ være, at terningen viser et lige antal øjne. Denne hændelse noteres $A=\{2,4,6\}$.

Eventuelt noget mere og hændelser.

\subsubsection{Sandsynligheden for en hændelse}
Sandsynligheden for at en hændelse forekommer er andelen af gange, hvor den givne hændelse, $A$, sker ved gentagelse af eksperiment, dette noteres som $P(A)$. Sandsynligheden for en hændelse er et tal mellem 0 og 1, $0 \le P(A) \le 1$. Summen af alle sandsynlighederne for udfaldsrummet er 1, $P(S)=1$, og derved indikerer en sandsynlighed tæt på 1 en sandsynlighed for denne hændelse.
\newline
Hvis der er lige stor sandsynlighed for alle udfald i udfaldsrummet kalde sandsynlighedsfordelingen uniform. Dette er tilfældet ved kast af mønt såvel som kast med terning.


\subsection{Stokastiske variable}
En stokastisk variabel, $X$, tildeler variable i udfaldsrummet en talværdi. Dette kunne være antallet af plat ved 5 kast med mønt eller summen af øjne ved flere kast med to terninger. Den stokastiske varibel fordeler sig efter en sandsynlighedfunktion, $f(X)$.

\subsubsection{Diskrete stokastiske variable}
Ved diskrete stokastiske variable antager $X$ kun hele værdier. Det giver eksempelvis ikke mening at tale om en halv plat eller halve øjne på terninger, hvorfor begge disse er eksempler på diskrete stokastiske variable.
Sandsynlighedsfunktionen for en diskret stokatisk variabel beskriver sandligheden for at variablen antager en given værdi, $f(x)=P(X=x)$. Dette kaldes også for \emph{massefunktionen}.
Herudfra kan variablens fordelingsfunktion, $F(x)$, bestemmes. Denne anvendes til at bestemme sandsynligheden for, at den stokastiske variabel antager en værdi lig med eller mindre end $x$, $F(x)=P(X \le x)=\sum_{t \le x}^{} f(t)$ for $-\infty < x < \infty$.

\paragraph{Middel} Middelværdien kaldes også for den forventede værdi, $E(X)$. Denne beregnes ved brug af sandsynlighedsfunktionen og er et vægtet gennemsnit, hvor hver mulig værdi for $X$ indgår med sin sandsynlighed, $f(x)$. Den forventede værdi beskriver det gennemsnitlige resultat ved mange gentagelser af samme eksperiment.
$$\mu=E(X)=\sum_{x}^{}x \cdot f(x)$$

\paragraph{Varians og standardafvigelse} Middelværdien alene kan ikke give en tilstrækkelig beskrivelse af data fordelingen. Man må også have beregreber, der beskriver dataets variabiliet. Den bedste størrelse til netop dette formål er varians, $\sigma ^2$. Variansen er den gennemsnitlige kvadrerede afstand til middelværdien og er givet ved formlen:
$$\sigma ^2=E[(X-\mu)^2]=\sum_{x}^{}(x-\mu)^2 \cdot f(x)$$

Kvadratroden af variansen kaldes standardafvigelsen.

\subsubsection{Kontinuerte stokastiske variable}

\subsection{Normalfordelingen}

\subsection{Chi i anden-fordelingen}

\subsection{Stikprøver}