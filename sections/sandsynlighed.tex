\section{Sandsynlighed}
Noget indledende tekst om sandsynlighed

\subsection{Udfaldsrum og hændelser}
Mængden af alle mulige udfald fra et statisk eksperiment kaldes for \emph{udfaldsrummet,  S}, et hvert udfald kalde et element eller medlem af udfaldsrummet. Det vil sige, at udfaldsrummet for kast med en mønt indeholder de to elementer; plat og krone.
En delmængde af udfaldsrummet kaldes en \emph{hændelse, A}, dette noteres som $A \subseteq S$. Ved kast med en terning kan en hændelse, $A$ være, at terningen viser et lige antal øjne. Denne hændelse noteres $A=\{2,4,6\}$.

Eventuelt noget mere og hændelser.

\subsubsection{Sandsynligheden for en hændelse}
Sandsynligheden for at en hændelse forekommer er andelen af gange, hvor den givne hændelse, $A$, sker ved gentagelse af eksperiment, dette noteres som $P(A)$. Sandsynligheden for en hændelse er et tal mellem 0 og 1, $0 \le P(A) \le 1$. Summen af alle sandsynlighederne for udfaldsrummet er 1, $P(S)=1$, og derved indikerer en sandsynlighed tæt på 1 en sandsynlighed for denne hændelse.
\newline
Hvis der er lige stor sandsynlighed for alle udfald i udfaldsrummet kalde sandsynlighedsfordelingen uniform. Dette er tilfældet ved kast af mønt såvel som kast med terning.


\subsection{Stokastiske variable}
En stokastisk variabel tildeler variable i udfaldsrummet en talværdi.
\paragraph{Middel}
\paragraph{Varians og standardafvigelse}

\subsubsection{Diskrete stokastiske variable}
\subsubsection{Kontinuerte stokastiske variable}

\subsection{Normalfordelingen}

\subsection{Chi i anden-fordelingen}

\subsection{Stikprøver}